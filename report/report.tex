% This file is based on the "sig-alternate.tex" V1.9 April 2009
% This file should be compiled with V2.4 of "sig-alternate.cls" April 2009

\documentclass{sig-alternate}

\usepackage{url}
\usepackage{color}
\usepackage{enumerate}
\usepackage{balance}
\permission{}
\CopyrightYear{2010}
%\crdata{0-00000-00-0/00/00}
\begin{document}

\title{Peer-to-Peer Pong}
\numberofauthors{3}
\author{
\alignauthor
Max Bogue
\alignauthor
Brian Gernhardt
\alignauthor
Aniket Sharma
}
\date{07 April 2011}
\maketitle

\begin{abstract}
	These days, even simple games are expected to have a multi-player
	version.  But once any program needs to deal with multiple machines,
	it must deal with the issues inherent in distributing programming.  To
	explore how these issues must be dealt with in even a simple game, we
	will implement a variant of Pong as a multiplayer game on multiple
	mobile devices.
\end{abstract}

\section{Overview}
\label{overview}

% Provide a roadmap for the remaining sections of the
% paper. For example, you can state that Section \ref{design
%   considerations} presents a discussion of the design issues
% you considered, and and section \ref{implementation}
% discusses how you went about implementing your project.

To explore distributed systems, we plan on implementing a distributed game.
In order to explore the problems behind the concept instead of spending
time implementing a complicated game, we intend to create a version of a
very simple and classic video game, Pong.  The objective of Pong is simple:
to keep a ball from falling off your side of the playing field.

To expand the game for a distributed setting, we alter the rules slightly.
Each player sees a portion of the total field.  The edges of their section
connect to other player's screen.  If the ball falls off the bottom of your
screen, it counts as a point against you.  The goal becomes having the
lowest score in the game.  To keep the game interesting with a large number
of players, we plan on having multiple balls bouncing around and to display
an overview of the entire game when there are no balls near your goal.

Section \ref{design considerations} discusses in more detail the design of
the game and systems involved.  Section \ref{implementation} discusses the
architecture of the code.  Section \ref{areas of work} contains the
problems we expect to find and our plans to deal with them.  Finally,
section \ref{current status} details the current status of the project.

\section{Design Considerations}
\label{design considerations}

% Use this section to describe the basic design of your project.

The project involves setting up a centralized server, which acts as a
reliable entity for initial game setup.  The server maintains the scores
and handles scenarios where players leave the game.  The server has the
authority of adding players to the game before the game starts and also
makes sure that the game functions within its rules.  It also has the
responsibility of keeping scores while the game is in progress.  The game
will also include additional features such as adding multiple balls in the
game and giving players the benefits of powerups and other real time
add-ons, thus making it scalable.  

The players co-ordinate with each other by sending constant messages about
the ball and the paddle position to rest of the players as well as the
server.  Thus, in case of a failure on any of the player devices the server
can make the necessary changes with the information it has, thereby
ensuring availability of the game to the rest of the players.  The actual
transfer of the messages between two players is peer to peer, thereby
increasing the speed and efficiency of the game.  The player who is about
to receive the ball will receive a message from the player where the ball
came from containing information about the location of the ball on the
screen, the angle and speed with which it is traveling ensuring uniform
game experience to the rest of the players

\section{Implementation}
\label{implementation}

% Use this section to describe the implementation of your
% project.

Our client code will be Java-based, to allow easy portability to the
Android and Blackberry platforms, as well as a desktop version.  The server
will also be written in Java in order to simplify development and reuse
some of the networking code.  The client interface will have to utilize the
various input methods for each platform.  Android users will be able to
touch the screen directly, BlackBerry users will use the trackball, and the
desktop user will have their usual input options.  We're hoping that the
design of the gameplay is simple enough that the user experience will be
relatively uniform across platforms.

Figure \ref{code design} depicts the high level structure of the code.  It
is split into four sections each encapsulating a portion of the system:
User Interface, Game Model, Server Communication and Client Communication.
This is motivated in not only the general concept of isolation, but also to
aid portability.  The User Interface code will likely need to be very
different on each platform and, at least on Blackberry, the networking code
as well.

The user interface is responsible for all interaction to the user,
including connecting to games, input to move the paddle and displaying the
ball's position with updates from the game engine.  The game engine
enforces all the rules of the game and models the limited ``physics'' of
the ball motion and bounces.  The two communications section handles all
issues with communicating with other systems, turning complex consensus
algorithms into simple updates of state.

The code interaction will be event driven, with the four sections each
either running in separate threads or being run as needed in response to
external events.  This is because each section needs to run in response to
different things and often with timing deadlines.  The user interface needs
to react to user input and update interactively, the game model needs to
update regularly and the communications portions need to respond to
external messages without causing timeouts.

\begin{figure}[htb]
	\label{code design}
	\begin{center}
		\includegraphics[width=3.25in]{UML.pdf}
		\caption{Code Design}
	\end{center}
\end{figure}

\section{Areas of Work}
\label{areas of work}

% Description of the problems to solve

\subsection{Game Setup}

The server handles an important issue of naming and service discovery.  The
centralized server ensures  that it is handling both the clients as well as
the backup servers at once. The game is setup by adding players and the
server ensures  that the game is played within the rules and is available
all the time. Along with maintaining the client list the server also
maintains the list of backup servers, always keeping them up to date about
the game situation in case the centralized server fails, this not only
handles availability of the game but also ensures consistency as far as the
game is concerned. The server has the authority of adding multiple balls
and also power ups, thus making the game scalable in its own way. The
issues that we handle as far as the project goes range from availability to
reliability by the use of an interesting algorithm.

\subsection{Coordinating Ball Position}

The players co-ordinate with each other by sending constant messages about
the ball and the paddle position to rest of the players as well as the
server. As the player interaction is peer to peer in order to increase
speed and efficiency, the issue about consensus regarding the ball
position, becomes the most important aspect of the game. We are going to
use Paxos\cite{lamport:paxos} for solving this issue.

The leader in this case is the player who currently has the ball. Thus the
use of Paxos not only ensures fairness in the game, but is highly useful in
cases of failures on client or server devices. The co-ordination between
the players as well as between players and the server makes our
architecture explore the benefits of both peer to peer as well as
centralized architecture scheme.  Reliability is ensured by Paxos, thus
making the gaming experience fair and real time.

\subsection{Client Connection Failures}

If a player drops from the game, this will be detected by each player
individually by a lack of response from game coordination messages. The
remaining players will then use Paxos to come to a consensus on whether or
not the player has lost their connection and should be dropped from the
game.

\subsection{Backup Servers}

Server failure will also be detected by a lack of response, and consensus
will again be reached using Paxos.  Since at this point in the design the
game requires the server to play, it cannot just be dropped.  Therefore the
clients will have to agree on a backup server to switch to.  The
involvement of backup servers ensures that there is no single point of
failure as far as communication is concerned.

When the clients originally connect to the server, they will be provided
with a list of additional servers that will be kept up to date with the
game state so they are available for hot swapping.  If a consensus is
reached that the server is inaccessible, the clients will connect to one of
the servers from this list.  In addition, the server itself could suggest a
switch to the backup server at game start in order to perform load
balancing.

\subsection{Scalability}

We plan on allowing the games to gracefully scale up to some undetermined
maximum.  We will need to be able to modify the game field in real time if
a player drops, so there is potential for a player to be able to join
mid-game as well.   Scaling of the system overall will be accomplished
through the concurrent use of multiple server machines, as well as trying
to keep the server-side portion of the game as minimal as possible.
Keeping the communication between the clients will inherently make scaling
to more players easier.

% \section{Mistakes Made and Corrected}
% \label{mistakes}

% Use this section to describe issues that the papers deal
% with that are neither common nor disagreements.

\section{Current Status and Future Work}
\label{current status}

% Use this section to describe the current status of your work
% and what else needs to be done.

A game physics engine and the beginning of the networking system have been
implemented, but the game is not yet playable.  We have rough
implementations of all four portions of the code as described previously,
but they are not yet well connected.  Our code repository is available
online at \texttt{https://github.com/maxbogue/BamPong}.

Further investigation into Blackberry's SDK has revealed a number of
inconsistencies with standard Java including the complete lack of the
\texttt{java.net} and \texttt{java.nio} packages.  This means that larger
portions of the application would have to be rewritten to run on a
Blackberry than originally anticipated.  In addition, problems with
obtaining and using the Blackberry SDK itself has already delayed
development.  While we have not yet abandoned the platform, its use has
become a significantly lower priority with development proceeding on
Android and a desktop application.

\bibliographystyle{abbrv}
\bibliography{report}
% You must have a proper ".bib" file
%  and remember to run:
% latex bibtex latex latex
% to resolve all references
\balance
\end{document}
