% This file is based on the "sig-alternate.tex" V1.9 April 2009
% This file should be compiled with V2.4 of "sig-alternate.cls" April 2009

\documentclass{sig-alternate}

\usepackage{url}
\usepackage{color}
\usepackage{enumerate}
\usepackage{balance}
\permission{}
\CopyrightYear{2010}
%\crdata{0-00000-00-0/00/00}
\begin{document}

\title{Peer-to-Peer Pong}
% \numberofauthors{3}
% \author{
% \alignauthor
% Max Bogue
% \alignauthor
% Brian Gernhardt
% \alignauthor
% Aniket Sharma
% }
\date{13 May 2011}
\maketitle

\begin{abstract}
	These days, even simple games are expected to have a multi-player
	version.  But once any program needs to deal with multiple machines, it
	must deal with the issues inherent in distributing programming.  To
	explore how these issues must be dealt with in even a simple game, we
	implemented a variant of Pong as a multiplayer game on multiple
	platforms.
\end{abstract}

\section{Overview}
\label{overview}

% Provide a roadmap for the remaining sections of the
% paper. For example, you can state that Section \ref{design
%   considerations} presents a discussion of the design issues
% you considered, and and section \ref{implementation}
% discusses how you went about implementing your project.

To obtain hands on experience with distributed systems, we implemented a
distributed game.  In order to explore the problems behind the concept
instead of spending time implementing a complicated game, we created a
version of a very simple and classic video game, Pong.  The objective of
Pong is simple: to keep a ball from falling off your side of the playing
field.

To expand the game for a distributed setting, we altered the rules
slightly.  Each player sees a portion of the total field.  The edges of
their section connect to other player's screen.  If the ball falls off the
bottom of your screen, it counts as a point against you.  The goal becomes
having the lowest score in the game.  To keep the game interesting with a
large number of players, we have multiple balls bouncing around.

This distributed nature lends itself to a peer-to-peer structure but we
included a central server to handle administrative tasks of starting games
and keeping score, as well as providing a central location for peers to
connect to in order to find each other.  In the style of modern video
games, we designed a system that can run on multiple platforms and
communicate via the Internet.

The organization of this paper is as follows.  Section \ref{motivation}
describes the motivations behind the design decisions we made.  Section
\ref{architecture} describes the overall architecture of the system and
section \ref{design} goes into more depth about each aspect of the design.
Section \ref{implementation} describes how we implemented the design and
section \ref{status} gives our current status.  We wrap up with the lessons
we learned in the course of the project in section \ref{lessons} and what
areas could be explored with additional work in section \ref{future}.


\section{Project Motivation}
\label{motivation}

We decided early on to create some form of game in order to make the result
entertaining in addition to educational.  In addition, the current
explosion of “apps” on mobile makes it possible to turn a project of this
nature into a commercial program.  The nature of the game helped make
several decisions about the form of the overall system.

In order to contrast different messaging styles, we implemented both
peer-to-peer and client-server architectures.  This allowed us to use
whichever mode was best for each task in the system.  The peer-to-peer
nature came naturally from the need to pass a ball from player to player,
but the existence of a central authority make tasks like finding other
players and adding new balls simple.  This mixture allowed us to explore
two methods of networking while actually making the implementation simpler.

Our final motivation was to create a program that ran across multiple
platforms.  This came out of the desire to mirror the “app” stores, where
the most popular programs are available on multiple platforms.  However, we
limited ourselves to platforms with Java compatibility so that we could try
to share as much code as possible.  This forced us to maintain a modular
system so that platform specific code could be kept separate while giving
us an appreciation for the work needed to do cross-platform work.


\section{Architecture}
\label{architecture}

Our system architecture is split into two sections: the server and the game
clients.  The server provides the centralized decision making while the
peer-to-peer nature of the clients keeps the server responsibilities small.
The server does add a single point of failure, so a backup needs to be
planned for.  Finally, in a complex system the needs of scalability needs
to be kept in mind.

\subsection{Client-Server}

Providing a centralized server gives a reliable entity for initial game
setup.  Having a single initial location for clients to connect to
simplifies the problem of discovery and creates an authority to give unique
identifiers to each client.  This naturally extends to having the server
handle the tasks of maintaining a list of open games and handling requests
for new players to join them.

The role of the server as central authority also plays a role in the course
of a game.  It can act as a scorekeeper, providing not only a single place
to store the value but also providing oversight to help keep misbehaving
clients from keeping the ball or penalizing their score if they refuse to
either pass the ball or report it as dropping within a certain timeout.

\subsection{Peer-to-Peer}

During a game, the players the players co-ordinate with each other by
sending constant messages about the ball position to rest of the players as
well as the server.  Thus, in case of a failure on any of the player
devices the server can make the necessary changes with the information it
has, thereby ensuring availability of the game to the rest of the players.
The actual transfer of the messages between two players is peer to peer,
thereby increasing the speed and efficiency of the game.  The player who is
about to receive the ball will receive a message from the player where the
ball came from containing information about the location of the ball on the
screen, the angle and speed with which it is traveling ensuring uniform
game experience to the rest of the players.

\subsection{Backup Servers}

Because the server has responsibility for scoring and re-adding balls once
they have dropped, the availability of the server is vital to the game.
When the clients originally connect to the server, they will be provided
with a list of additional servers that will be kept up to date with the
game state so they are available for hot swapping. In addition, the server
itself could suggest a switch to the backup server at game start in order
to perform load balancing.

\subsection{Scalability}

To maintain scalability, much of the work is kept away from the central
server.  Keeping a majority of the communication between clients only makes
scaling simpler as they only need to scale based on the number of players
in a game.  If the work the server needs to do for any particular game
message is kept simple, it should be able to handle a large number of games
without issue.  The backup server system implies the ability to scale by
adding multiple servers and switching between them based on load.


\section{Design}
\label{design}

\begin{figure}[htb]
	\label{code design}
	\begin{center}
		\includegraphics[width=3.25in]{UML.pdf}
		\caption{Code Design}
	\end{center}
\end{figure}

The code can be divided into four major parts: user interaction, game
engine, server communication and peer communication.  The details of each
part is described in the next section, but the way they need to interact is
described here.  In particular, the tasks of setting up a game, running the
game, passing the ball, and client and server failures are addressed.

\subsection{Game Setup}

The server handles an important issue of naming and service discovery.  The
centralized server ensures that it is handling both the clients as well as
the backup servers at once. The game is setup by adding players and the
server ensures that the game is played within the rules and is available
all the time. Along with maintaining the client list the server also
maintains the list of backup servers, always keeping them up to date about
the game situation in case the centralized server fails, this not only
handles availability of the game but also ensures consistency as far as the
game is concerned. The server has the authority of adding multiple balls,
thus making the game scalable in its own way. The issues that we handle as
far as the project goes range from availability to reliability by the use
of an interesting algorithm.

\subsection{Running the Game}

Each player is responsible for determining the motion of the ball while it
is within their section of the game.  The unique “physics” of the game is
contained within the game engine, which is solely responsible for the
motion of the paddle and ball.  This requires cooperation with all other
sections of code.  It must update the GUI to display the game to the user
and update the paddle location in response to user input.  It must accept
and pass balls off to other clients.  And finally, it must notify the
server when the ball falls off the bottom of the screen.  

\subsection{Passing the Ball}

The players co-ordinate with each other by sending constant messages about
the ball and the paddle position to rest of the players as well as the
server. As the player interaction is peer to peer in order to increase
speed and efficiency, the issue about consensus regarding the ball
position, becomes the most important aspect of the game. We decided ing to
use Paxos\cite{lamport:paxos} for solving this issue.

The leader in this case is the player who currently has the ball. Thus the
use of Paxos not only ensures fairness in the game, but is highly useful in
cases of failures on client or server devices. The co-ordination between
the players as well as between players and the server makes our
architecture explore the benefits of both peer to peer as well as
centralized architecture scheme.  Reliability is ensured by Paxos, thus
making the gaming experience fair and real time.

\subsection{Client Connection Failures}

If a player drops from the game, this will be detected by each player
individually by a lack of response from game coordination messages. The
remaining players will then use Paxos to come to a consensus on whether or
not the player has lost their connection and should be dropped from the
game.  If the consensus is that the player is still playing, attempts to
reconnect can be made and another vote called for if it fails.

\subsection{Server Connection Failures}

Server failure can be detected by a lack of response to a client request,
and the players can vote to achieve a group consensus on if the server has
failed.  The involvement of backup servers ensures that there is no single
point of failure as far as communication is concerned.   If a consensus is
reached that the server is inaccessible, the clients will connect to one of
the servers from this list.


\section{Implementation}
\label{implementation}

Our client code is written in Java, to allow easy portability to the
Android and Blackberry platforms, as well as a desktop version.  The server
was also be written in Java in order to simplify development and reuse some
of the networking code.  The client interface also has to utilize the
various input methods for each platform.  Android users can touch the
screen directly, BlackBerry users use the trackball, and the desktop user
has their usual mouse and keyboard options.  The design of the gameplay is
kept simple enough that the user experience is relatively uniform across
platforms.

\begin{figure}[htb]
	\label{code layout}
	\begin{center}
		\includegraphics[width=3.25in]{Implementation.pdf}
		\caption{Code Layout}
	\end{center}
\end{figure}

% TODO: Change to describe actual code
% TODO: Add note about threads

Figure \ref{implementation} depicts the structure of the code.  Each box
indicates a Java class.  The GUI classes are platform dependent, with an
implementation in Swing on the desktop and using Activities on Android.
All other code is shared across the platforms.

The user interface is responsible for all interaction to the user,
including connecting to games, input to move the paddle and displaying the
ball's position with updates from the game engine.  The game engine
enforces all the rules of the game and models the limited ``physics'' of
the ball motion and bounces.  The two communications section handles all
issues with communicating with other systems, turning complex consensus
algorithms into simple updates of state.

The code interaction is event driven, with the communication classes and
the Engine running in separate threads in addition to the main thread and
any GUI threads started by the platform libraries.  This is because each of
these need to run in response to different things and often with timing
deadlines.  The user interface needs to react to user input and update
interactively, the game model needs to update regularly and the
communications portions need to respond to external messages without
causing timeouts.

All of these portions are coordinated by the Client class, which forwards
and translates events from one subsystem to another.  Ball drop messages
are caught from the engine and forwarded to ServerCommunication to notify
the server.  Ball passing messages are similarly caught and forwarded to
PeerCommunication.  Messages from the server and peers are forwarded to the
engine or user as appropriate.  This weak coupling between the components
insulates each section from the others and allows the Engine to simply note
a ball has left the field without any knowledge of the other peers or
server.

Despite the PeerCommunication and Server classes needing to communicate
with multiple destinations, they are contained to a single thread by using
\texttt{java.nio.channels.Selector}s.  The selector contains a reference to
every socket that may send messages and the thread sleeps on it until a
message arrives.  The thread then dispatches the message as quickly as
possible and returns to waiting on the selector.

\subsection{Server}

The server is a single threaded application with no user interface,
although it does output log messages.  This is so it can be started on a
server and left to run for an extended period of time.  Although it does
share some networking code with the client, it does not use the game engine
or communication classes.  The server is intended to be as small as
possible and maintain only the state needed to respond to clients.  This is
so that a single server can serve many games without requiring heavy
hardware.

\subsection{Desktop}

The Desktop client uses the Swing library to interact with the user.  The
user is prompted for a nickname and server address when the application is
started.  The application then starts the common peer and server
communication classes and ties them together with a client object.  It then
displays an interface to create, join, and start games by interacting with
the client.  It must act as a listener to ServerCommunications however, so
that it can be notified when a game is going to start which may be
initiated by another client.  When the game is started, it displays the
game field and all further interaction is handled by the client and engine.

\subsection{Android}

The Android user interface consists of three Activities:  Start, GUI, and
BamPong.   These handle similar responsibilities to the Swing application,
but are structured according to the expectations of the Android system.

The Start activity simply displays a prompt for player name and server
address.  When these are entered and the user presses the connect button,
this information is passed to the GUI activity.  On creation the GUI
activity uses the common communication classes.  Any errors are displayed
as alert dialogs to the user and the Start activity is displayed again.

The GUI activity gives the client the options to create game, join game,
cancel game and start game respectively.  Again here we have implemented
user alerts to counter wrong information entered.  Each of the options is a
button, which when clicked tries to create a client object which allows the
client to create game, join game start or cancel respectively.  In case of
creation of the game the client has to wait for other players to join the
game.  Once that happens a peer to peer connection between the players is
established via the peer communication class.  The join game button when
clicked displays a list of games for the client to select from, this list
of games itself is a new activity.  Similarly the create game button when
clicked switches to a new activity called CreateGame which gives the user
the option to create the game.

The BamPong activity is started when notification arrives from the server
that the game is starting.  Here the actual game field, the balls and
paddles are drawn.  The drawing is done by using built-in Android graphics
libraries.  The Activity constantly interacts with the engine for the game
physics in real time.  Also the interaction with the server and Client is
happening for the transfer information of the balls being passed from the
top of the screen.


\section{Current Status}
\label{status}

The earliest change to our original plans was the decision to not develop
on Blackberry.  After several days of e-mail problems to get registered, we
discovered that their developement environment was a customized version of
Eclipse that would not interact with other systems such as the Android SDK.
In addition, although Blackberry uses Java it uses a custom set of
libraries instead of the standard Java libraries which meant that much
larger sections of our code would need to be rewritten for Blackberry.

In response the dropping the Blackberry platform, we expanded the desktop
client.  Most of the original desktop code was written to test new features
rapidly.

\begin{itemize}
	\item \textbf{Desktop Expanded:} Originally, the desktop version was a
		simple test platform but was expanded to become a full game client.
	\item \textbf{Networking:} Paxos turned out to be a more complicated
		algorithm than originally expected.  When the implementation had
		problems, we went back to a more simple (but less failure
		resistant) message passing method.
\end{itemize}

The Android platform is mostly functional, but due to the disconnect method
of development and indirect nature of its APIs, it has proved difficult to
debug.  We successfully developed the GUI and managed to connect to the
server and the peers in the Game.  We were successful in making the game
engine compatible with android along with the networking we implemented for
desktop.  The networking was mostly compatible but we could not fix the
connection drop bug on Android which would drop the connection after the
game was started.  The android client is able to create, cancel, join and
start a game, due to the connection issues with the server it is not able
to sustain the game for long.  The graphics on Android works fine
displaying and connecting the required entities.  The Android is compatible
and successfully connects to other Android and desktop clients.  It
successfully inputs user data and channels it to the appropriate receivers.
It successfully displays user alerts.  We had a few minor bugs that were
dropping the connection to a game after few seconds on the Android client,
but we are sure we should be able to fix it in the future.

\section{Lessons Learned}
\label{lessons}

Our team did not have much experience with any major section of the
project: Android, networking and game physics.  It took much longer than we
had planned to get basic implementations.The biggest lesson learned was
knowing or practicing basic application development concepts before we
implemented the Distributed application.Knowing or practicing coding about
simple networking and life cycle of an Android app  would have helped
definitely to code the networking part and the game physics part on Android
better.Making simple android applications and testing them on android
devices should be a prerequisite before developing a Distributed Android
application.We should have made ourselves well versed with the life cycle
of a simple Android app and then should have migrated on to the Distributed
Version of the life cycle rather than jumping on the Distributed life cycle
in the first place.We left learning the Android concepts late and should
have done that before we started implementing.We were exploring concepts as
we were coding ,but we should have had an entire picture of all important
concepts of Android known before we started coding.This would have given us
more options and time.Android developing environment was the essential
component for this project.We should have explored the eclipse
functionalities of Android before we started coding instead of doing it
simultaneously.Debugging on Android is complex and we should have made
ourselves familiar with that with more practice.

We each worked on a major section of code, learning the techniques required
and developing in pieces.  We left integrating the sections until late in
the project.  We should have planned and integrated the codes much earlier
than we did,this would have given us more time to solve problems that
resulted after the integration of the major sections.  We tried to
implement the networking part ourselves instead of using the inbuilt
libraries.  The use of inbuilt networking libraries on Android as well as
Java would have made our job simpler and would have saved us time given
that we had to implement the project in ten weeks.

We realized it very late in our implementation that even Simple Distributed
Algorithms were complicated.We tried to implement Paxos as a consensus
algorithm without realizing the complexity and the amount of time it would
take to implement it.``Fast Paxos'' (Lamport, 2005) describes the Paxos
algorithm in 4 steps over three pages.  ``Paxos for System Builders''
(Kirsh and Amir, 2008) gives complete pseudo code for the same algorithm
and takes 10 pages.  The lesson learned here was determining early in your
project whether it would be feasible and practical  to implement  complex
concepts  within a given time frame or not.

\section{Future Work}
\label{future}

\begin{itemize}
	\item \textbf{Server Failure:} Backup servers were never implemented
	\item \textbf{Consensus Algorithms:} Paxos seemed to be far more
		complex than required for this project.  Other forms of voting and
		data propagation could be investigated.
	\item \textbf{Additional Platforms:} While Blackberry was never
		developed due to lack of time, the modular nature of the code
		should allow the differences to be abstracted away.
\end{itemize}

\section{Conclusion}

\bibliographystyle{abbrv}
\bibliography{report}
% latex bibtex latex latex
\balance

\end{document}
