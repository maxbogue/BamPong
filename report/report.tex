% This file is based on the "sig-alternate.tex" V1.9 April 2009
% This file should be compiled with V2.4 of "sig-alternate.cls" April 2009

\documentclass{sig-alternate}

\usepackage{url}
\usepackage{color}
\usepackage{enumerate}
\usepackage{balance}
\permission{}
\CopyrightYear{2010}
%\crdata{0-00000-00-0/00/00}
\begin{document}

\title{Peer-to-Peer Pong}
\numberofauthors{3}
\author{
\alignauthor
Max Bogue
\alignauthor
Brian Gernhardt
\alignauthor
Aniket Sharma
}
\date{06 December 2010}
\maketitle
% \begin{abstract}
% \end{abstract}

\section{Overview}
\label{overview}

% Provide a roadmap for the remaining sections of the
% paper. For example, you can state that Section \ref{design
%   considerations} presents a discussion of the design issues
% you considered, and and section \ref{implementation}
% discusses how you went about implementing your project.

To explore distributed systems, we plan on implementing a distributed game.
In order to explore the problems behind the concept instead of spending
time implementing a complicated game, we intend to create a version of a
very simple and classic video game, Pong.  The objective of Pong is simple:
to keep a ball from falling off your side of the playing field.

To expand the game for a distributed setting, we alter the rules slightly.
Each player sees a portion of the total field.  The edges of their section
connect to other player's screen.  If the ball falls off the bottom of your
screen, it counts as a point against you.  The goal becomes having the
lowest score in the game.  To keep the game interesting with a large number
of players, we plan on having multiple balls boucing around and to display
an overview of the entire game when there are no balls near your goal.

In this document we will describe the design of the system, our
implementation plan, and the problems we expect to need to solve.  As our
project continues, we will update this report with more detailed
information and references to the sources used to overcome the problems
described at the end of this document.

\section{Design Considerations}
\label{design considerations}

% Use this section to describe the basic design of your project.

The project involves setting up a centralized server,which acts as a
reliable entity for initial game setup.  The server maintains the scores
and handles scenarios where players leave the game.  The server has the
authority of adding players to the game before the game starts and also
makes sure that the game functions within its rules.  It also has the
responsibility of keeping scores while the game is in progress.  The game
will also include additional features such as adding multiple balls in the
game and giving players the benefits of powerups and other real time
add-ons,thus making it scalable.  

The players co-ordinate with each other by sending constant messages about
the ball and the paddle position to rest of the players as well as the
server.  Thus,in case of a failure on any of the player devices the server
can make the necessary changes with the information it has,thereby ensuring
availability of the game to the rest of the players.  The actual transfer
of the messages between two players is peer to peer,thereby increasing the
speed and efficiency of the game.  The player who is about to receive the
ball will receive a message from the player where the ball came from
containing information about the location of the ball on the screen,the
angle and speed with which it is travelling ensuring uniform game
experience to the rest of the players


\section{Implementation}
\label{implementation}

% Use this section to describe the implementation of your
% project.

Our client code will be Java-based, to allow easy portability to the
Android and Blackberry platforms, as well as a desktop version.  The server
may also be written in Java, but that is less definite as that doesn't
require the same ease of re-use.  The client interface will have to utilize
the various input methods for each platform.  Android users will be able to
touch the screen directly, BlackBerry users will use the trackball, and the
desktop user will have their usual input options.  We're hoping that the
design of the gameplay is simple enough that the user experience will be
relatively uniform across platforms.


\section{Areas of Work}
\label{areas of work}

% Description of the problems to solve

There will be a number of interesting problems to overcome in the
development of this game.  Game field setup will be interesting; that is,
how it is decided which players are adjacent to which others.  This bears
some small resemblance to the Byzantine Generals Problem, if we decide to
try to do it peer-to-peer.  Related is what should happen if a player drops
from the game; should the field be rearranged to exclude them?  That seems
like the most elegant solution.  A fault-tolerant server setup will also be
a challenge; hopefully one that this class will teach us how to solve.
Finally, a way of tracking and guaranteeing the ball's position against
cheating will also play a crucial roll.


% \section{Mistakes Made and Corrected}
% \label{mistakes}

% Use this section to describe issues that the papers deal
% with that are neither common nor disagreements.

% \section{Current Status and Future Work}
% \label{current status}

% Use this section to describe the current status of your work
% and what else needs to be done.


% \bibliographystyle{abbrv}
% \bibliography{report}
% You must have a proper ".bib" file
%  and remember to run:
% latex bibtex latex latex
% to resolve all references
\balance
\end{document}








