% This file is based on the "sig-alternate.tex" V1.9 April 2009
% This file should be compiled with V2.4 of "sig-alternate.cls" April 2009

\documentclass{sig-alternate}

\usepackage{url}
\usepackage{color}
\usepackage{enumerate}
\usepackage{balance}
\permission{}
\CopyrightYear{2010}
%\crdata{0-00000-00-0/00/00}
\begin{document}

\title{Project Report (Rename To Fit Your Project As Needed)}
\numberofauthors{1}
\author{
\alignauthor
No Such Names
}
\date{06 December 2010}
\maketitle
\begin{abstract}
  The abstract should be one or two paragraphs that
  summarize your paper. Abstracts are read independently
  from the rest of the paper so you cannot cite your paper
  or other papers in it. Study other abstracts in the papers
  you are reading to understand what an abstract should
  really means. Write the abstract in third person.
\end{abstract}

\section{Overview}
\label{overview}

Use this section to present an overview of the project.

Provide a roadmap for the remaining sections of the
paper. For example, you can state that Section \ref{design
  considerations} presents a discussion of the design issues
you considered, and and section \ref{implementation}
discusses how you went about implementing your project.

{\bf Note: This specific file is a generic template so the
  section titles may not fit your specific needs so feel
  free to change them as needed.}

\section{Design Considerations}
\label{design considerations}

Use this section to describe the basic design of your project.

\section{Implementation}
\label{implementation}

Use this section to describe the implementation of your
project.

\section{Mistakes Made and Corrected}
\label{mistakes}

Use this section to describe issues that the papers deal
with that are neither common nor disagreements.

\section{Current Status and Future Work}
\label{current status}

Use this section to describe the current status of your work
and what else needs to be done.

The next subsection is meant to provide you with some help
in dealing with figures, tables and citations.

\subsection*{Tables, Figures, and Citations/References}

Tables, figures, and citations/references in technical
documents need to be presented correctly. As many students
are not familiar with using these objects, here is a quick
guide extracted from the ACM style guide.

\begin{table}
\centering
\caption{Feelings about Issues}
\begin{tabular}{|l|r|l|} \hline
Flavor&Percentage&Comments\\ \hline
Issue 1 &  10\% & Loved it a lot\\ \hline
Issue 2 &  20\% & Disliked it immensely\\ \hline
Issue 3 &  30\% & Didn't care one bit\\ \hline
Issue 4 &  40\% & Duh?\\ \hline
\end{tabular}
\end{table}


First, note that figures in the term paper must be original,
that is, created by the student: please do not cut-and-paste
figures from any other paper you have read. Second, if you
do need to include figures, they should be handled as
demonstrated here. State that Figure \ref{sample graphic} is
a simple illustration used in the ACM Style sample
document. Figures are never below or above the
text. Incidentally, in proper technical writing (for reasons
beyond the scope of this discussion), table captions are
above the table and figure captions are below the figure.

\begin{figure}[htb]
\label{sample graphic}
\begin{center}
\includegraphics[width=2in]{fly.jpg}
\caption{A sample black \& white graphic (JPG).}
\end{center}
\end{figure}

Finally, citing documents needs to be done properly too. For
example, a paper by Mic Bowman, Saumya K. Debray, and Larry
L. Peterson could be cited as Bowman, Debray, and Peterson
\cite{bowman:reasoning}. A set of papers could collectively
be cited as the literature in this area consists of several
interesting papers
\cite{braams:babel,clark:pct,herlihy:methodology}.

The list of all references will be generated in ACMRef
standard style using the \LaTeX{}/BibTeX. Note that you
need to first the following sequence to get the paper
compiled correctly:

\begin{enumerate}
\item {\tt latex} {\em termpaper}
\item {\tt bibtex} {\em termpaper}
\item {\tt latex} {\em termpaper}
\item {\tt latex} {\em termpaper}
\end{enumerate}

\bibliographystyle{abbrv}
\bibliography{report}
% You must have a proper ".bib" file
%  and remember to run:
% latex bibtex latex latex
% to resolve all references
\balance
\end{document}








